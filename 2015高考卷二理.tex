\documentclass[UTF8]{ctexart}
\usepackage{amsmath,empheq}[fleqn]
\usepackage{amssymb}

\begin{document}
\title{2016年普通高等学校招生全国统一考试
    (课标全国卷\uppercase\expandafter{\romannumeral2}).\\
    \textbf{理数}}
\author{Wang}
\date{\today}
\maketitle

\section{选择题}
\subsection{12.设函数$f'(x)$是奇函数$f(x)(x \in \mathbb{R})$的导函数,$f(-1)=0$,
当$x>0$时,$xf'(x)-f(x)<0$,则使得$f(x)>0$成立的$x$的取值范围是}
A.$(-\infty,-1)\cup(0,1)\ B.(-1,0)\cup(1,+\infty)\ 
C.(-\infty,-1)\cup(-1,0)\ D.(0,1)\cup(1,+\infty)$\\

解:\\
$f(x)$为奇函数,则$f(0)=0,f(-x)=-f(x),f(-1)=0$则$f(1)=0$\\
又$xf'(x)-f(x)<0$形如$g(x)=\frac{f(x)}{x}$的导数$g'(x)=\frac{xf'(x)-f(x)<0}{x^2}$\\
则$x>0时,g'(x)<0$,则$g(x)$在$(x>0)$上单调递减\\
则$x>1$时,$f(x)<f(1)=0;x\in(0,1)$时,$f(x)>f(1)>0$\\
又$g(-x)=\frac{f(-x)}{-x}=\frac{f(x)}{x}=g(x)$则$g(x)$为偶函数\\
则$\forall x\in(-1,0),g(x)>0;\forall x\in(-\infty,-1),g(x)<0$\\
且$\forall x<0,f(x),g(x)$异号,$\forall x>0,f(x),g(x)$同号\\
故$f(x)>0$的解集为$(-\infty,-1)\cup(0,1)$,故选A
\section{解答题}
\subsection{20.本题12分}
已知椭圆$C:9x^2+y^2=m^2(m>0)$,直线$l$不过原点$O$且不平行于坐标轴,$l$与$C$
有两个交点$A,B$,线段$AB$的中点为$M$.\\
\begin{enumerate}
    \item 证明:直线$OM$的斜率与$l$的乘积为定值;
    \item 若$l$过点$(\frac{m}{3},m)$,延长线段$OM$与$C$交于点$P$,四边形$OAPB$能否为
平行四边形?若能,求此时$l$的斜率;若不能,说明理由
\end{enumerate}
1.证明:\\
(点差法)设$A(x_1,y_1),B(x_2,y_2)$,则有:\\
$k_l=\frac{y_1-y_2}{x_1-x_2},
k_{OM}=\frac{\frac{1}{2}(y_1+y_2)}{\frac{1}{2}(x_1+x_2)}=\frac{(y_1+y_2)}{(x_1+x_2)}$
\begin{empheq}[left=\empheqlbrace]{align}
        &9x_1^2+y_1^2=m^2\\
        &9x_2^2+y_2^2=m^2
\end{empheq}
$(1)-(2)$得:\\
\[
\begin{aligned}
    &9(x_1+x_2)(x_1-x_2)+(y_1+y_2)(y_1-y_2)=0 \\
    &\frac{y_1+y_2}{x_1+y_2}\cdot \frac{y_1-y_2}{x_1-x_2}=-9
\end{aligned}
\]
即$k_l \cdot k_{OM}=-9$,证毕.\\
2.解:\\
$l$过点$(\frac{m}{3},m)$,且$l$不过圆点,不与坐标轴垂直,可设
\[ l:y=k(x-\frac{m}{3})+m=kx+(1-\frac{k}{3})m\]
对于椭圆$C$上的点$(x,y)$,有$x\leq \frac{m}{3},y\leq m$\\
故要使$l$与$C$有两个交点,需有$k>0$且$k\neq 3$\\
要使四边形$OAPB$为平行四边形,需使$M$同时也是$OP$中点\\
联立$l$与$C$:
\begin{empheq}[left=\empheqlbrace]{align}
    &y=kx+(1-\frac{k}{3})m\\
    &9x^2+y^2=m^2
\end{empheq}
\[\begin{aligned}
    &9x^2+[k^2x^2+(1-\frac{k}{3})^2m^2+2mk(1-\frac{k}{3})x]=m^2\\
   &(9+k^2)x^2+2mk(1-\frac{k}{3})x+m^2(\frac{k^2}{9}-\frac{2}{3}k)=0\\
   \Rightarrow &x_1+x_2=\frac{2mk(\frac{k}{3}-1)}{9+k^2}
\end{aligned}\]
又由(1)知$k_{OM}=-\frac{9}{k}$,再联立$l_{OP}$与C:\\
\begin{empheq}[left=\empheqlbrace]{align}
    &y=-\frac{9}{k}x\\
    &9x^2+y^2=m^2
\end{empheq}
\[\begin{aligned}
    &9x^2+\frac{81}{k^2}x^2=m^2\\
    \Rightarrow &x_p=\frac{\pm mk}{3\sqrt{k^2+9}}
\end{aligned}\]
由$OM=MP \Rightarrow x_p=2x_m=x_1+x_2$:
\[\begin{aligned}
    &\frac{\pm mk}{3\sqrt{k^2+9}}=\frac{2mk(\frac{k}{3}-1)}{9+k^2}\\
    &\frac{\frac{k}{3}-1}{\sqrt{9+k^2}}=\pm \frac{1}{6}\\
    &\frac{k^2}{9}+1-\frac{2k}{3}=\frac{9+k^2}{36}\\
    &4k^2+36-24k-9-k^2=0\\
    &3k^2-24k+27=0\\
    &k^2-8k+9=0\\
    \Rightarrow k&=\frac{8\pm \sqrt{64-4\times 9}}{2}\\
            k&= 4\pm \sqrt{7}
\end{aligned}\]
且$4\pm \sqrt{7}>0,\neq 3$\\
故$OAPB$可以是平行四边形,此时$l$的斜率为$4+\sqrt{7}$或$4-\sqrt{7}$

\subsection{21.本题12分}
设函数$f(x)=e^{mx}+x^2-mx.$\\
\begin{enumerate}
    \item 证明:$f(x)$在$(-\infty,0)$单调递减,在$(0,+\infty)$单调递增;
    \item 若对于任意$x_1,x_2\in [-1,1]$,都有$|f(x_1)-f(x_2)|\leq e-1,$求$m$的取值范围.
\end{enumerate}
1.证明:\\
\[
\begin{aligned}
    f'(x)&=me^{mx}+2x-m\\
    &=m(e^{mx}-1)+2x
\end{aligned}\]
\begin{enumerate}
    \item $m> 0$时:\\
    若$x\geq 0:m(e^{mx}-1)\geq 0,2x\geq 0,f'(x)\geq 0$;\\
    若$x<0:m(e^{mx}-1)<0,2x<0,f'(x)<0$;\\
    \item $m<0$时:\\
    若$x\geq 0:m(e^{mx}-1)\geq 0,2x\geq 0,f'(x)\geq 0$;\\
    若$x<0:m(e^{mx}-1)>0,2x<0,f'(x)<0$;\\
\end{enumerate}
故:$f(x)$在$(-\infty,0)$单调递减,在$(0,+\infty)$单调递增;证毕.\\
2.解:\\
由(1)知,$\forall x \in [1,1],f(x)$在$x=0$处取得最小值\\
在$x=1$或$x=-1$处取得最小值\\
故\[
\begin{aligned}
    &|f(x_1)-f(x_2)|\leq e-1\\
    \Leftrightarrow &\left\{
        \begin{aligned}
            f(1)-f(0)\leq e-1\\
            f(-1)-f(0)\leq e-1
        \end{aligned}
    \right.
    \Leftrightarrow &\left\{
        \begin{aligned}
            e^m-m\leq e-1\\
            e^{-m}+m\leq e-1
        \end{aligned}
    \right.
\end{aligned}\]
记函数$g(x)=e^x-x$,则$g(m)=e^m-m,g(-m)=e^{-m}+m$\\
则$g'(x)=e^x>0$故$g(x)$单调递增\\
又$g(1)=e-1$,故$e^m-m\leq e-1 \Leftrightarrow g(m)\leq g(1)\Leftrightarrow m\leq 1$\\
类似的,有$e^{-m}+m\leq e-1 \Leftrightarrow g(-m)\leq g(1)
\Leftrightarrow -m\leq 1 \Leftrightarrow m\geq -1$\\
综上,满足题设的m的取值范围为$[-1,1]$

\end{document}