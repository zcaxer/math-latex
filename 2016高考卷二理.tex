\documentclass[UTF8]{ctexart}
\usepackage{amsmath}[fleqn]
\usepackage{amssymb}

\begin{document}
\title{2016年普通高等学校招生全国统一考试
    (课标全国卷\uppercase\expandafter{\romannumeral2}).\\
    \textbf{理数}}
\author{Wang}
\date{\today}
\maketitle

\section{解答题}
\subsection{20.已知椭圆E:$\frac{x^2}{t}+\frac{y^2}{3}=1$的焦点在$x$轴上,
    A是E的左顶点,斜率为$k(k>0)$的直线交E于A,M两点,点N在E上,MA$\perp$NA.}
\begin{enumerate}
    \item 当$t=4$,$|AM|=|AN|$时,求$\triangle AMN$的面积.
    \item 当$2|AM|=|AN|$时,求$k$的取值范围.
\end{enumerate}
解:\\
1:\\
$t=4$时,$A(-2,0)$\\
又$|AM|=|AN|$,则M,N在以A为圆心的圆上\\
且M,N在E上,故M,N关于$x$轴对称\\
又MA$\perp$NA\\
则$\angle MAO= \angle NAO=\frac{\pi}{4} $\\
故直线MA:$y=x+2$\\
将$ y=x+2 $代入$\frac{x^2}{4}+\frac{y^2}{3}=1$得: \\
\[
    \begin{aligned}
         & 3x^2+4(x^2+4x+4)-12=0 \\
         & 7x^2+16x+4=0          \\
         & (7x+2)(x+2)=0         \\
    \end{aligned}
\]
解得$ x=-2$(舍去),或$x=-\frac{2}{7}$\\
此时$y=-\frac{2}{7}+2=\frac{12}{7}$\\
则$S\triangle AMN=\frac{1}{2}\times [-\frac{1}{2}-(-2)]\times \frac{12}{7}\times 2=\frac{144}{49}$\\

2:\par
由E方程知,$A(-\sqrt{t},0)$,又$k>0$且易知k存在\\
可设直线$AM:y=k(x+\sqrt{t})$\\
即$x=\frac{y}{k}-\sqrt{t}$\\
带入E方程得:
\[\begin{aligned}
        \frac{(\frac{y}{k}-\sqrt{t})^2}{t}+\frac{y^2}{3}=1  \\
        3(\frac{y^2}{k^2}-\frac{2y\sqrt{t}}{k}+t)+ty^2-3t=0 \\
        (3+tk^2)y^2-6ky\sqrt{t}=0                           \\
    \end{aligned}\]
$y=0$(舍去),或$y=\frac{6k\sqrt{t}}{3+tk^2}$\\
则$$\begin{aligned}
        |AM| & =\sqrt{1+\frac{1}{k^2}}(\frac{6k\sqrt{t}}{3+tk^2}) \\
             & =\frac{6\sqrt{t(1+k^2)}}{3+tk^2}
    \end{aligned}$$
类似的,用$-\frac{1}{k}$代换$k$可得:\\
\[|AN|=\frac{6k\sqrt{t(1+k^2)}}{3k^2+t} \]
由$2|AM|=|AN|$则:
\[\begin{aligned}
        \frac{2}{3+tk^2}=\frac{k}{3k^2+t} \\
        6k^2+2t=3k+tk^3                   \\
        (k^3-2)t=3k(2k-1)
    \end{aligned}  \]
$k^3=2$时,等式不成立,故有:
\[t=\frac{3k(2k-1)}{k^3-2}  \]
由E的焦点在x轴上,有:
\[\begin{aligned}
        t>3                           \\
        \frac{3k(2k-1)}{k^3-2} >3     \\
        \frac{-k^3+2K^2-k+2}{K^3-2}>0 \\
        \frac{(k-2)(k^2+1)}{k^3-2}<0  \\
        \Leftrightarrow (k-2)(k^3-2)<0
    \end{aligned}\]
又$k>0$,
故k的取值范围为$(\sqrt[3]{2},2)$

\subsection{21.本题12分}
\begin{enumerate}
    \item 讨论函数$f(x)=\frac{x-2}{x+2}e^x$的单调性,并证明当$x>0$时,$(x-2)e^x+x+2>0$;
    \item 证明:当$a\in [0,1)$时,函数$g(x)=\frac{e^x-ax-a}{x^2}(x>0)$有最小值.设$g(x)$的
          最小值为$h(a)$,求函数$h(a)$的值域.
\end{enumerate}
1.证明:\\
\[ \begin{aligned}
        f'(x) & =[\frac{(x+2)-(x-2)}{(x+2)^2}+\frac{x-2}{x+2}]e^x \\
              & =\frac{x^2e^x}{(x+2)^2}\geq 0
    \end{aligned}
\]
且仅当$x=0$时,$f'(x)=0$\\
则$f(x)$在$(-\infty,-2),(-2,\infty)$单调递增\\
故有当$x>0$时,$f(x)>f(0)=-1$\\
即:\[\begin{aligned}
        \frac{x-2}{x+2}e^x>-1 \\
        (x-2)e^x+x+2>0
    \end{aligned}    \]
证毕\\
2.解:\\
\[\begin{aligned}
    g'(x)&=\frac{(e^x-a)x^2-2x(e^x-ax-a)}{x^4}\\
    &=\frac{(x-2)e^x+a(x+2)}{x^3}
\end{aligned} \]
注意,分子与1中结论类似,分母$x^3>0(x>0)$
受此启发,变形得:
\[ g'(x)=\frac{(x+2)[f(x)+a]}{x^3} \]
则$g'(x)$与$f(x)+a$同号 \\
由1知$f(x)+a$在$[0,+\infty)$上单调递增 \\
且$f(0)+a=a-1<0,f(2)+a=a\geq 0$\\
则必有唯一点$x_0 \in (0,2]$使$f(x)+a=0$\\
且当$x<x_0$时,$g'(x)<0;x>x_0$时,$g'(x)>0$\\
且$x_0$满足:
\[\begin{aligned}
    f(x_0)+a=0 \\
    \frac{(x_0-2)e^x_0}{x_0+2}+a=0 \\
    (x_0-2)e^x_0=-a(x_0+2)\\
    e^x_0=\frac{-a(x_0+2)}{x_0-2}\\
\end{aligned}\]
则$g(x)$在$x_0$处取得最小值$h(a)$为:
\[\begin{aligned}
        h(a)=g(x_0)&=\frac{e^x_0-ax_0-a}{x_0^2} \\
        &=\frac{-(x_0+2)-(x_0+1)(x_0-2)}{(x_0-2)x_0^2}a\\ 
        &=\frac{-ax_0^2}{(x_0-2)x_0^2}  \\
        &=\frac{-a}{x_0-2}  
\end{aligned}\]
(思考:要求$h(a)$的值域,需知其单调性,需求导数,
而表达式中同时含有$a$以及与$a$相关的中间变量$x_0$,高中阶段无法求导,故需再次变形。
观察$x_0$与$a$的关系式后易得)
\[h(a)=\frac{e^x_0}{x_0+2}\]
(理解:$h(a)$中并未闲事出现自变量$a$,而$x_0$的值与$a$相关,即$x_0=f_1(a)$,
相当于$x_0$作为中间变量,$f_2(x_0)=\frac{e^x_0}{x_0+2}$,$f_2(x_0)$的定义域
为$f_1(a)$的值域,$f_2(x_0)$的值域与$h(a)$的值域相同)
\[\begin{aligned}
    f_2'(x_0)&=\frac{e^x_0(x_0+2)-e^x_0}{(x_0+2)^2}\\
    &=\frac{e^x_0(x_0+1)}{(x_0+2)^2>0}
    \end{aligned} \]
故$f_2(x_0)$单调递增,而$x_0 \in (0,2]$\\
$f_2(0)=\frac{1}{2},f_2(2)=\frac{e^2}{4}$\\
故$f_2(x_0)$的值域为$(\frac{1}{2},\frac{e^2}{4}]$\\
也即$h(a)$的值域为$(\frac{1}{2},\frac{e^2}{4}]$
\end{document}