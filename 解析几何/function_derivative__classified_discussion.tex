\documentclass{ctexbeamer}

\usepackage{standalone}
\usepackage{amsmath,amssymb,enumitem}
\renewcommand\parallel{\mathrel{/\mskip-2.5mu/}}
\setCJKmainfont{SimHei} 
\setsansfont{Times New Roman}

\begin{document}
\title{导函数为二次函数的分类讨论}
\author{王骝维}
\institute{不惑面试}

\frame{\titlepage}

\frametitle{1}
\begin{frame}

    $f(x)=x-a\ln{x}+\frac{1+a}{x}$
    \begin{enumerate}[label=(\arabic*)]
        \item   若 $a=1$,求 $f(x)$ 在 $x \in [1,3]$的最小值;
        \item   求 $f(x)$ 的单调区间
        \item   若 $\exists x \in [1,e]$,使得 $f(x)<0$,求 $a$ 的取值范围
    \end{enumerate}

\end{frame}

\begin{frame}
    (1) $a=1,f'(x)
        \begin{aligned}[t]
             & =x-\frac{1}{x}-\frac{2}{x^2} \\
             & =\frac{x^2-x-2}{x^2}         \\
             & =\frac{(x+1)(x-2)}{x^2}      \\
        \end{aligned}$
    故在 $x=2$处取得最小值为 $f(2)=2-\ln{2}+1=3-\ln{2}$\\
    $f(1)=3,f(3)=\frac{7}{3}-\ln{\frac{2}{3}}$\\
    故在 $x \in [1,3]$上的最大值和最小值分别为 $f(1)=3,f(2)=3-\ln{2}$
\end{frame}


\begin{frame}
    (1):\\
    $a=1,f'(x)
        \begin{aligned}[t]
             & =x-\frac{1}{x}-\frac{2}{x^2} \\
             & =\frac{x^2-x-2}{x^2}         \\
             & =\frac{(x+1)(x-2)}{x^2}      \\
        \end{aligned}$
    故在 $x=2$处取得最小值为 $f(2)=2-\ln{2}+1=3-\ln{2}$\\
    故 $x \in [1,3]$时, $f(x)$ 最小值为 $f(2)=3-\ln{2}$
\end{frame}

\begin{frame}
    (2):\\
    $f'(x)
        \begin{aligned}[t]
             & =x-\frac{a}{x}-\frac{1+a}{x^2} \\
             & =\frac{x^2-ax-1-a}{x^2}        \\
             & =\frac{(x+1)(x-1-a)}{x^2}      \\
        \end{aligned}$
    由 原函数定义域为 $x\in(0,+\infty)$\\
    $x>-1$则$ f(x)$ 与 $x-a-1$ 同号\\
    a>1时,
\end{frame}
\end{document}