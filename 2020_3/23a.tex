\documentclass[class=ctexart,crop=false]{standalone}

\usepackage{amsmath,amssymb,enumitem,empheq,tkz-euclide,
diagbox,wrapfig,pgfplots,geometry}
%\geometry{a4paper,scale=0.9}
\pgfplotsset{compat=newest}
%\usepgfplotslibrary{external}
%\tikzexternalize
\renewcommand\parallel{\mathrel{/\mskip-2.5mu/}}

\newcommand\px{\mathrel{/\mkern-5mu/}}  %平行
\newcommand\pxeq{\mathrel{\vcenter{     %平行且等于
\ialign{\hfil##\hfil\crcr
$\scriptstyle\px\!$\crcr
\noalign{\nointerlineskip\vskip1pt}$=$\crcr}}}}

%\setCJKmainfont{SimSun}       %设置西文字体为times new roman
%\setCJKsansfont{SimSun}             %设置中文字体为宋体
%\setCJKmonofont{STKaiti}
%\setsansfont{TeX Gyre Termes}            %设置typewriter family中文字体为楷体
%\setmonofont{TeX Gyre Termes}

\usetikzlibrary{calc,intersections,through,backgrounds,patterns}
\newcounter{para}
\newcommand\mypara{\par\refstepcounter{para}\thepara.\space}%设置typewriter family西文字体为times new roman
\newcommand*\circled[1]{\tikz[baseline=(char.base)]{
            \node[shape=circle,draw,inner sep=1pt] (char) {#1};}}

\newcommand{\rnum}[1]{\romannumeral #1}
\newcommand{\RNum}[1]{\uppercase\expandafter{\romannumeral #1\relax}}
\begin{document}
\begin{enumerate}[label=(\arabic*)]
    \item 证明:\\
    由 $(a+b+c)^2=a^2+b^2+c^2+2ab+2bc+2ac=0\\$
    得: $ab+bc+ca=-\frac{a^2+b^2+c^2}{2}$\\
    又由 $abc=1$,故 $a,b,c \neq 0$\\
    则: $a+b+c<0$
    \item 证明:(法一)\\
    易知 $a,b,c$中有且仅有2个为负数,一个为正数,不妨设该正数为 $a$\\
    则 $max\{a,b,c\}$即为a;\\
    由 $a+b+c=0,\Rightarrow b=-a-c$\\
    又由$abc=1 \Rightarrow \begin{aligned}[t]
        -ac(a+c)=1\\
        ac^2+a^2c+1=0
    \end{aligned} $\\
    将上式视为 $a$为参数关于 $c$的二次方程,若要 $c$存在,则必有:\\
    $a^4-4a\geqslant 0$\\
    则:$(a^3-4)\geqslant 0$\\
    又 $a\neq 0$,故有 $a \geqslant \sqrt[3]{4}$\\
    法二:\\
    $b=\frac{1}{ac},\begin{aligned}[t]
        & \text{由} a+b+c=1\\
       & \Leftrightarrow a=(\frac{-1}{ac}-c) \geqslant 2\sqrt{\frac{1}{a}}\\
        &\Leftrightarrow a- 2\sqrt{\frac{1}{a}}\geqslant 0\\
        &\Leftrightarrow a^\frac{3}{2}- \geqslant 2\\
        &\Leftrightarrow a \geqslant \sqrt[3]{4}\\
    \end{aligned} $\\
    法三:$a=\frac{1}{bc},a^2=(b+c)^2,\\
    a^3=\frac{(b+c)^2}{bc}=\frac{b^2+c^2+2bc}{bc}\geqslant \frac{2bc+2bc}{bc}=4$
\end{enumerate}

\end{document}