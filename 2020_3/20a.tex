\documentclass[class=ctexart,crop=false]{standalone}

\usepackage{amsmath,amssymb,enumitem,empheq,tkz-euclide,
diagbox,wrapfig,pgfplots}
\pgfplotsset{compat=newest}
\renewcommand\parallel{\mathrel{/\mskip-2.5mu/}}

\newcommand\px{\mathrel{/\mkern-5mu/}}  %平行
\newcommand\pxeq{\mathrel{\vcenter{     %平行且等于
\ialign{\hfil##\hfil\crcr
$\scriptstyle\px\!$\crcr
\noalign{\nointerlineskip\vskip1pt}$=$\crcr}}}}

%\setCJKmainfont{SimSun}       %设置西文字体为times new roman
%\setCJKsansfont{SimSun}             %设置中文字体为宋体
%\setCJKmonofont{STKaiti}
%\setsansfont{TeX Gyre Termes}            %设置typewriter family中文字体为楷体
%\setmonofont{TeX Gyre Termes}

\usetikzlibrary{calc,intersections,through,backgrounds,patterns}
\newcounter{para}
\newcommand\mypara{\par\refstepcounter{para}\thepara.\space}%设置typewriter family西文字体为times new roman
\newcommand*\circled[1]{\tikz[baseline=(char.base)]{
            \node[shape=circle,draw,inner sep=1.3pt] (char) {#1};}}
\begin{document}
解:\\
\begin{enumerate}[label=(\arabic*)]
    \item
    由题意:$e^2=\frac{25-m^2}{25}=\frac{15}{16}\\
   \Rightarrow  m^2=\frac{25}{16}$\\
   故 $ C:\frac{x^2}{25}+\frac{y^2}{\frac{25}{16}}=1 $\\
    \item 
    \begin{tikzpicture}
      \begin{axis}[
        unit vector ratio*=1 1 1 ,
        axis x line = middle,
        axis y line = middle,
        xlabel      = {$x$},
        ylabel      = {$y$},
        xmin=-6.5, xmax=7.5,
        ymin=-2, ymax=9,
        legend pos=outer north east,
        legend style={draw=none},
      ]
        \addplot +[no markers,
          raw gnuplot,
          thick,
          black,
          empty line = jump % not strictly necessary, as this is the default behaviour in the development version of PGFPlots
        ] gnuplot {
          set xrange [-6:6];
          set yrange [-2:2];
          set contour base;
          set cntrparam levels discrete 0.003;
          unset surface;
          set view map;
          set isosamples 500;
          set samples 500;
          splot x^2/25 + 16*y^2/25 - 1;
        };
        \node[label={120:{A}},circle,fill,inner sep=1pt] (A) at (axis cs:-5,0) {};
        \node[label={90:{B}},circle,fill,inner sep=1pt] (B) at (axis cs:5,0) {};
        \node[label={90:{$P_1$}},circle,fill,inner sep=1pt] (P1) at (axis cs:-3,1) {};
        \node[label={90:{$P_2$}},circle,fill,inner sep=1pt] (P2) at (axis cs:3,1) {};
        \node[label={0:{$Q_1$}},circle,fill,inner sep=1pt]  (Q1) at (axis cs:6,8) {};
        \node[label={0:{$Q_2$}},circle,fill,inner sep=1pt] (Q2) at (axis cs:6,2) {};
        \draw (6,-2)--(6,9);
        \draw (P1)--(Q1)--(A)--(P1)--(B)--(Q1);
        \draw (P2)--(Q2)--(A)--(P2)--(B)--(Q2);
      \end{axis}
      \end{tikzpicture}\\
      易知 $B:(5,0)$,设 $P:(x_1,y_1), Q:(6,m)$\\
      不妨设 $m > 0$,则 $y_1>0$\\
      则 $BQ:y=\frac{m-0}{6-5}(x-5)=m(x-5),|BQ|=\sqrt{1+m^2}$\\
      由 $BQ \perp BP,BP:y=-\frac{1}{m}(x-5)\\$
      则 $\begin{aligned}[t]
        BP&=\sqrt{(y_P-y_B)^2+(x_P-x_B)^2}\\
        &=\sqrt{(1+m^2)(y_1)^2}\\
      \end{aligned}$\\
      又: $|BP|=|BQ|\Rightarrow(y_1)^2=1$\\
      则 $y_1=1$\\
      代入 $C$ 得 :$x^2+16=25\Rightarrow x=\pm 3$\\
      故 $P_1:(-3,1),P_2:(3,1)$\\
      则 $k_{BP_1}=\frac{1}{-3-5}=-\frac{1}{m}\Rightarrow m=8$\\
      同理 $k_{BP_2}=\frac{1}{3-5}=-\frac{1}{m}\Rightarrow m=2$\\
      则:$P_1Q_1:y=\frac{8-1}{6+3}(x+3)+1=\frac{7}{9}x+\frac{10}{3}\Rightarrow 7x-9y+30=0$\\
      同理:$P_2Q_2:y=\frac{2-1}{6-3}(x-3)+1=\frac{1}{3}x\Rightarrow x-3y=0 $\\
      则 $ \begin{aligned}[t]
        S\triangle AP_1Q_1&=\frac{1}{2}|P_1Q_1|d_{A-P_1Q_1}\\
        &=\frac{1}{2}\sqrt{(8-1)^2+(6+3)^2}\times\frac{|-5\times 7 -0+30|}{\sqrt{7^2+(-9)^2}}\\
        &=\frac{1}{2}\sqrt{49+81}\times \frac{5}{\sqrt{49+81}} \\
        &=\frac{5}{2}
      \end{aligned} $\\
      同理 $ \begin{aligned}[t]
        S\triangle AP_2Q_2&=\frac{1}{2}|P_2Q_2|d_{A-P_2Q_2}\\
        &=\frac{1}{2}\sqrt{(2-1)^2+(6-3)^2}\times\frac{|-5\times 1 -0+0|}{\sqrt{1^2+3^2}}\\
        &=\frac{1}{2}\sqrt{1+9}\times \frac{5}{\sqrt{10}} \\
        &=\frac{5}{2}
      \end{aligned} $\\
      综上 $\triangle APQ $的面积为 $\frac{5}{2}$\\
      \end{enumerate}

   


\end{document}