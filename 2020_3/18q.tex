\documentclass[class=ctexart,crop=false]{standalone}

\usepackage{amsmath,amssymb,enumitem,empheq,tkz-euclide,
diagbox,wrapfig,pgfplots}
\pgfplotsset{compat=newest}
\renewcommand\parallel{\mathrel{/\mskip-2.5mu/}}

\newcommand\px{\mathrel{/\mkern-5mu/}}  %平行
\newcommand\pxeq{\mathrel{\vcenter{     %平行且等于
\ialign{\hfil##\hfil\crcr
$\scriptstyle\px\!$\crcr
\noalign{\nointerlineskip\vskip1pt}$=$\crcr}}}}

%\setCJKmainfont{SimSun}       %设置西文字体为times new roman
%\setCJKsansfont{SimSun}             %设置中文字体为宋体
%\setCJKmonofont{STKaiti}
%\setsansfont{TeX Gyre Termes}            %设置typewriter family中文字体为楷体
%\setmonofont{TeX Gyre Termes}

\usetikzlibrary{calc,intersections,through,backgrounds,patterns}
\newcounter{para}
\newcommand\mypara{\par\refstepcounter{para}\thepara.\space}%设置typewriter family西文字体为times new roman
\newcommand*\circled[1]{\tikz[baseline=(char.base)]{
            \node[shape=circle,draw,inner sep=1.3pt] (char) {#1};}}
\begin{document}
某学生兴许小组随机调查了某市100天中每天的空气质量等级
和当天到某公园锻炼的人次,整理数据得到下表(单位:天):\\
\begin{tabular}{|c|c|c|c|}
    \hline
    \diagbox {空气质量等级}{锻炼人次} &$[0,200]$&$(200,400]$&$(400,600]$\\
    \hline
    1(优) & 2& 16&25\\ 
    \hline
    2(良) &5&10&12\\
    \hline
    3(轻度污染) &6&7&8\\
    \hline
    4(重度污染) &7&2&0\\
    \hline
\end{tabular}
\begin{enumerate}[label=(\arabic*)]
    \item 分别估计该市一天的空气质量等级为$1,2,3,4$的概率;
    \item 求一天中到该公园锻炼的平均人次的估计值(同一组中的
    数据用该区间的中点值为代表);
    \item 若某天的空气质量等级为1或2,则称这天“空气质量好”;
    若某天的空气质量等级为3或4,则称这天“空气质量不好”.根据
    所给数据,完成下面的$2\times 2$列联表,并根据列联表,判
    断是否有$95\%$的把握认为一天中到该公园锻炼的人次与该市当
    天的空气质量有关?
\end{enumerate}
\begin{tabular}{|c|c|c|}
    \hline
        &人次 $\leqslant 400 $ &人次 $> 400 $\\
        \hline
    空气质量好 & \quad & \\
    \hline
    空气质量不好 & &\\
    \hline
\end{tabular} \\
附: $K^2=\frac{n(ad-bc)^2}{(a+b)(c+d)(a+c)(b+d)}$,
\begin{tabular}{c|ccc}
    $P(K^2 \geqslant k)$ &$0.050$&$0.010$&$0.001$\\
    \hline
    $k$&$3.841$&$6.635$&$10.828$\\
\end{tabular}

\end{document}