\documentclass[class=ctexart,crop=false]{standalone}

\usepackage{amsmath,amssymb,enumitem,empheq,tkz-euclide,
diagbox,wrapfig,pgfplots,geometry}
%\geometry{a4paper,scale=0.9}
\pgfplotsset{compat=newest}
%\usepgfplotslibrary{external}
%\tikzexternalize
\renewcommand\parallel{\mathrel{/\mskip-2.5mu/}}

\newcommand\px{\mathrel{/\mkern-5mu/}}  %平行
\newcommand\pxeq{\mathrel{\vcenter{     %平行且等于
\ialign{\hfil##\hfil\crcr
$\scriptstyle\px\!$\crcr
\noalign{\nointerlineskip\vskip1pt}$=$\crcr}}}}

%\setCJKmainfont{SimSun}       %设置西文字体为times new roman
%\setCJKsansfont{SimSun}             %设置中文字体为宋体
%\setCJKmonofont{STKaiti}
%\setsansfont{TeX Gyre Termes}            %设置typewriter family中文字体为楷体
%\setmonofont{TeX Gyre Termes}

\usetikzlibrary{calc,intersections,through,backgrounds,patterns}
\newcounter{para}
\newcommand\mypara{\par\refstepcounter{para}\thepara.\space}%设置typewriter family西文字体为times new roman
\newcommand*\circled[1]{\tikz[baseline=(char.base)]{
            \node[shape=circle,draw,inner sep=1pt] (char) {#1};}}

\newcommand{\rnum}[1]{\romannumeral #1}
\newcommand{\RNum}[1]{\uppercase\expandafter{\romannumeral #1\relax}}
\begin{document}
\begin{enumerate}[label=(\arabic*)]
    \item 解:\\
    由题意得:$f'(\frac{1}{2})=0$\\
    而$f'(x)=3x^2+b\\$
    故$f'(\frac{1}{2})=\frac{3}{4}+b=0\\
    \Rightarrow b=-\frac{3}{4}$
    \item 证明:(法一)\\
    由题意,$\exists x_0 \in [-1,1]$,使$f(x_0)=x_0^3-\frac{3}{4}x_0+c=0$\\
    则$c=-x_0^3+\frac{3}{4}x_0$\\
    则 $\begin{aligned}[t]
        f(x)&=x^3-x_0^3-\frac{3}{4}x+\frac{3}{4}x_0\\
        &=(x-x_0)(x^2+x_0x+x_0^2)-\frac{3}{4}(x-x_0)\\
        &=(x-x_0)(x^2+x_0x+x_0^2-\frac{3}{4})\\
    \end{aligned} $\\
    则 $f(x)$ 的其他零点必为方程 $x^2+x_0x+x_0^2-\frac{3}{4}=0$的根\\
    $$x=\frac{-x_0 \pm \sqrt{x_0^2-4(x_0^2-\frac{3}{4})}}{2}=\frac{-x_0\pm \sqrt{3-3x_0^2}}{2}$$\\
    不妨设 $x_0=\sin \theta(\theta \in [-\frac{\pi}{2},\frac{\pi}{2}])$,\\
    则 $\begin{aligned}[t]
        x&=-\frac{1}{2}\sin \theta \pm \frac{\sqrt{3}}{2} \cos \theta\\
        &=-\sin (\theta \pm \frac{\pi}{3})\\
    \end{aligned}$\\
    故 $|x| \leqslant1$\\
    故$f(x)$  所有零点的绝对值都不大于1\\
    法二:\\
    $f'(x)=0 \Rightarrow 3x^2-\frac{3}{4}=0 \Rightarrow x=\pm \frac{1}{2}$\\
    故 $x\in (-\infty,-\frac{1}{2}], f(x)$单调递增,
    $x\in (-\frac{1}{2},\frac{1}{2}), f(x)$单调递减,
    $x\in [-\frac{1}{2},\infty), f(x)$单调递增\\
    $f(-1)=c-\frac{1}{4},f(-\frac{1}{2})=c+\frac{1}{4},
    f(\frac{1}{2})=c-\frac{1}{4},f(1)=c+\frac{1}{4}$\\
    若有一个零点$\in [-1,1]$,则 $c-\frac{1}{4}\leqslant 0,c+\frac{1}{4}\geqslant 0$\\
    故 $x \in (-\infty,-1),x < f(-1)=c-\frac{1}{4}<0$\\
    $x \in (1,-\infty),x > f(1)=c+\frac{1}{4}>0$\\
    故 $f(x)$所有零点均在 $[-1,1]$中,即 $f(x)$  所有零点的绝对值都不大于1\\
\end{enumerate}
\end{document}