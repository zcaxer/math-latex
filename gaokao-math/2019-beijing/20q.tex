\documentclass[class=ctexart,crop=false]{standalone}

\usepackage{amsmath,amssymb,enumitem,empheq,tkz-euclide,
diagbox,wrapfig,pgfplots,geometry}
%\geometry{a4paper,scale=0.9}
\pgfplotsset{compat=newest}
%\usepgfplotslibrary{external}
%\tikzexternalize
\renewcommand\parallel{\mathrel{/\mskip-2.5mu/}}

\newcommand\px{\mathrel{/\mkern-5mu/}}  %平行
\newcommand\pxeq{\mathrel{\vcenter{     %平行且等于
\ialign{\hfil##\hfil\crcr
$\scriptstyle\px\!$\crcr
\noalign{\nointerlineskip\vskip1pt}$=$\crcr}}}}

%\setCJKmainfont{SimSun}       %设置西文字体为times new roman
%\setCJKsansfont{SimSun}             %设置中文字体为宋体
%\setCJKmonofont{STKaiti}
%\setsansfont{TeX Gyre Termes}            %设置typewriter family中文字体为楷体
%\setmonofont{TeX Gyre Termes}

\usetikzlibrary{calc,intersections,through,backgrounds,patterns}
\newcounter{para}
\newcommand\mypara{\par\refstepcounter{para}\thepara.\space}%设置typewriter family西文字体为times new roman
\newcommand*\circled[1]{\tikz[baseline=(char.base)]{
            \node[shape=circle,draw,inner sep=1pt] (char) {#1};}}

\newcommand{\rnum}[1]{\romannumeral #1}
\newcommand{\RNum}[1]{\uppercase\expandafter{\romannumeral #1\relax}}
\begin{document}
(本小题13分)\\
已知数列$\{a_n\}$,从中选择第 $i_1$ 项、第 $i_2$ 项、\dots、第 $i_m$ 项 $(i_1<i_2<\cdots<i_m')$,
若 $a_{i_1}<a_{i_2}<\cdots<a_{i_m}$,则称新数列 $a_{i_1},a_{i_2},\cdots,a_{i_m}$ 为$\{a_n\}$的长度为 $m$ 的
递增子列.规定数列 $\{a_n\}$ 的任意一项都是 $\{a_n\}$ 的长度为1的递增子列.
\begin{enumerate}[label=(\Roman*)]
    \item 写出数列 $1,8,3,7,5,6,9$ 的一个长度为4的递增子列; 
    \item 已知数列 $\{a_n\}$ 的长度为 $p$ 的递增子列的末项的最小值为 $a_{m_0}$,长度为 $q$ 的递增子列的末项的最
    小值为 $a_{n_0}$.若 $p<q$ 求证: $a_{m_0}<a_{n_0}$;
    \item 设无穷数列 $\{a_n\} $ 的各项均为正整数,且任意两项均不相等.若 $\{a_n\}$ 的长度为 $s$ 的递增子列
    的末项的最小值为 $2s-1$,且长度为 $s$末项为 $2s-1$ 的递增子列恰有 $2^{s-1}$ 个 $(s=1,2,\cdots)$,求数列 $\{a_n\}$ 
    的通项公式 
\end{enumerate}
\end{document}