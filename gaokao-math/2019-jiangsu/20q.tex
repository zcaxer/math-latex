\documentclass[class=ctexart,crop=false]{standalone}

\usepackage{amsmath,amssymb,enumitem,empheq,tkz-euclide,
diagbox,wrapfig,pgfplots}
\pgfplotsset{compat=newest}
\renewcommand\parallel{\mathrel{/\mskip-2.5mu/}}

\newcommand\px{\mathrel{/\mkern-5mu/}}  %平行
\newcommand\pxeq{\mathrel{\vcenter{     %平行且等于
\ialign{\hfil##\hfil\crcr
$\scriptstyle\px\!$\crcr
\noalign{\nointerlineskip\vskip1pt}$=$\crcr}}}}

%\setCJKmainfont{SimSun}       %设置西文字体为times new roman
%\setCJKsansfont{SimSun}             %设置中文字体为宋体
%\setCJKmonofont{STKaiti}
%\setsansfont{TeX Gyre Termes}            %设置typewriter family中文字体为楷体
%\setmonofont{TeX Gyre Termes}

\usetikzlibrary{calc,intersections,through,backgrounds,patterns}
\newcounter{para}
\newcommand\mypara{\par\refstepcounter{para}\thepara.\space}%设置typewriter family西文字体为times new roman
\newcommand*\circled[1]{\tikz[baseline=(char.base)]{
            \node[shape=circle,draw,inner sep=1.3pt] (char) {#1};}}
\begin{document}
    定义首项为1且公比为正数的等比数列为”M-数列”.
    \begin{enumerate}[label=(\arabic*)]
        \item 已知等比数列 $\{a_n\}(n\in N^*) $满足: $a_2a_4=a_5,a_3-4a_2+4a_1=0$
        求证:数列 $\{a_n\} $为”M-数列”
        \item 已知数列 $\{b_n\}(n\in N*) $满足 $:b_1=,\frac{1}{S_1}=
        \frac{2}{b_n}-\frac{2}{b_{n+1}}$,其中 $S_n $为数列 $\{b_n\} $的
        前 $n$项和.
        \begin{enumerate}[label=\protect\circled{\arabic*}]
            \item 求数列 $\{b_n\} $的通项公式;
            \item 设 $m$为正整数,若存在”M-数列” $\{c_n\}(n\in N*), $
            对任意正整数 $k$,当 $k\leqslant m$时,都有 $c_k\leqslant b_k \leqslant c_{k+1}$
            成立,求 $m$ 的最大值.
        \end{enumerate}
    \end{enumerate}
\end{document}