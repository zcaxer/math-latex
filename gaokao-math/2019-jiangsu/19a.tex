\documentclass[class=ctexart,crop=false]{standalone}

\usepackage{amsmath,amssymb,enumitem,empheq,tkz-euclide,
diagbox,wrapfig,pgfplots,geometry}
%\geometry{a4paper,scale=0.9}
\pgfplotsset{compat=newest}
%\usepgfplotslibrary{external}
%\tikzexternalize
\renewcommand\parallel{\mathrel{/\mskip-2.5mu/}}

\newcommand\px{\mathrel{/\mkern-5mu/}}  %平行
\newcommand\pxeq{\mathrel{\vcenter{     %平行且等于
\ialign{\hfil##\hfil\crcr
$\scriptstyle\px\!$\crcr
\noalign{\nointerlineskip\vskip1pt}$=$\crcr}}}}

%\setCJKmainfont{SimSun}       %设置西文字体为times new roman
%\setCJKsansfont{SimSun}             %设置中文字体为宋体
%\setCJKmonofont{STKaiti}
%\setsansfont{TeX Gyre Termes}            %设置typewriter family中文字体为楷体
%\setmonofont{TeX Gyre Termes}

\usetikzlibrary{calc,intersections,through,backgrounds,patterns}
\newcounter{para}
\newcommand\mypara{\par\refstepcounter{para}\thepara.\space}%设置typewriter family西文字体为times new roman
\newcommand*\circled[1]{\tikz[baseline=(char.base)]{
            \node[shape=circle,draw,inner sep=1pt] (char) {#1};}}

\newcommand{\rnum}[1]{\romannumeral #1}
\newcommand{\RNum}[1]{\uppercase\expandafter{\romannumeral #1\relax}}
\begin{document}
    \begin{enumerate}[label=(\arabic*)]
        \item 由题意 $f(x)=(x-a)^3,f(4)=(4-a)^3=8\Rightarrow a=2$
        \item 由题意 $f(x)=(x-a)(x-b)^2\\
            f'(x)=(x-b)^2+2(x-a)(x-b)=(x-b)(3x-2a-b)$\\
            故 $\frac{2a+b}{3}$则 $a,b$只能为$-3$和 $3$\\
            若 $a=-3,b=3$时 $\frac{2a+b}{3}=-1$(舍去)\\
            若 $a=3,b=-3$时 $\frac{2a+b}{3}=1$(符合题意)\\
            故 $f(x)=(x-3)(x+3)^2,f'(x)=(x+3)(3x-3)$\\
            $x<-3,f'(x)>0;-3\leqslant x \leqslant 1,f'(x)\leqslant 0;x>1,f'(x)>0;$\\
            故在 $x=1$处取得极小值 $f(1)=-2*16=-32$            
        \item 由题意 $f(x)=x(x-b)(x-1)=(x^2-x)(x-b)=x^3-(b+1)x^2+bx\\
        f'(x)=3x^2-2(b+1)x+b$\\
        $f'(x)=0$时 $3x^2-2(b+1)x+b=0\\$
        易知$f(x)$在$x_0=\frac{2(b+1)-\sqrt{4(b+1)^2-4\cdot 3b}}{6}
        =\frac{b+1-\sqrt{b^2-b+1}}{3}$处取得极大值\\
        $x_0^2=\frac{2b^2+b+2-2(b+1)\sqrt{b^2-b+1}}{9}, 
        x_0^3=\frac{4b^3+3b^2+3b+4-(4b^2+5b+4)\sqrt{b^2-b+1}}{27}$\\
        $f(x_0)=\frac{-2b^3+3b^2+3b-2+(2b^2-2b+2)\sqrt{b^2-b+1}}{27}$
    \end{enumerate}
\end{document}