\documentclass[class=ctexart,crop=false]{standalone}

\usepackage{amsmath,amssymb,enumitem,empheq,tkz-euclide,
diagbox,wrapfig,pgfplots}
\pgfplotsset{compat=newest}
\renewcommand\parallel{\mathrel{/\mskip-2.5mu/}}

\newcommand\px{\mathrel{/\mkern-5mu/}}  %平行
\newcommand\pxeq{\mathrel{\vcenter{     %平行且等于
\ialign{\hfil##\hfil\crcr
$\scriptstyle\px\!$\crcr
\noalign{\nointerlineskip\vskip1pt}$=$\crcr}}}}

%\setCJKmainfont{SimSun}       %设置西文字体为times new roman
%\setCJKsansfont{SimSun}             %设置中文字体为宋体
%\setCJKmonofont{STKaiti}
%\setsansfont{TeX Gyre Termes}            %设置typewriter family中文字体为楷体
%\setmonofont{TeX Gyre Termes}

\usetikzlibrary{calc,intersections,through,backgrounds,patterns}
\newcounter{para}
\newcommand\mypara{\par\refstepcounter{para}\thepara.\space}%设置typewriter family西文字体为times new roman
\newcommand*\circled[1]{\tikz[baseline=(char.base)]{
            \node[shape=circle,draw,inner sep=1.3pt] (char) {#1};}}
\begin{document}
\begin{enumerate}[label=(\Roman*)]
    \item $g(x)=e^x({\cos{x}-\sin{x}})=-\sqrt{2}e^x\sin{(x-\frac{\pi}{4})}$\\
    故 $f(x)$的递增区间为 $[2k\pi+\frac{\pi}{4},2k\pi+\frac{5\pi}{4}]$\\
    故 $f(x)$的递减区间为 $[2k\pi-\frac{3\pi}{4},2k\pi+\frac{\pi}{4}]$
    \item 记$h(x)=f(x)+g(x)(\frac{\pi}{2}-x)$\\
    则 $h'(x)=g(x)+g'(x)(\frac{\pi}{2}-x)-g(x)\\
    =-2e^x\sin{x}(\frac{\pi}{2}-x)\\
    x \in [\frac{\pi}{4},\frac{\pi}{2}]$时,$h'(x)\leqslant 0$\\
    故 $h(x)$单调递减,
    且 $h(\frac{\pi}{2})=0$\\
    故此时 $h(x)\geqslant 0$
    \item 
    $x_n$为 $f(x)-1=0$的零点,即 $e^{x_n}\cos{x_n}=1$\\
    $x \in (2n\pi+\frac{\pi}{4},2n\pi+\frac{\pi}{2})$
    则$x-2n\pi \in (\frac{\pi}{4},\frac{\pi}{2})$\\
    $x_n$越大$e^x_n$也越大,则 $\cos{x_n}$越小,则 $x_n-2n\pi$越大\\
    $f(x_n-2n\pi)=e^{x_n-2n\pi}\cos{(x_n-2n\pi)}=e^{-2n\pi}$\\
    由 \RNum{2} 得 $f(x-2n\pi)+g(x_n-2n\pi)(\frac{\pi}{2}-x+2n\pi)\geqslant 0$\\
    考察 $g(x),g'(x)=-2e^x\sin{x},x\in (\frac{\pi}{4},\frac{\pi}{2})$时, $g'(x)<0$\\
    故 $g(x)$单调递减, $n>1$时,$g(x_n-2n\pi)< g(x_0)<0$\\
    故 $\frac{\pi}{2}-x+2n\pi \leqslant \frac{-e^{-2n\pi}}{g(x_n-2n\pi)}<\frac{e^{-2n\pi}}{-g(x_0)}=\frac{e^{-2n\pi}}{e^{x_0}(sinx_0-cosx_0)}<\frac{e^{-2n\pi}}{\sin{x_0}-\cos{x_0}}$\\
    x
\end{enumerate}
\end{document}