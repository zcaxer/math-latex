\documentclass[class=ctexart,crop=false]{standalone}

\usepackage{amsmath,amssymb,enumitem,empheq,tkz-euclide,
diagbox,wrapfig,pgfplots,geometry}
%\geometry{a4paper,scale=0.9}
\pgfplotsset{compat=newest}
%\usepgfplotslibrary{external}
%\tikzexternalize
\renewcommand\parallel{\mathrel{/\mskip-2.5mu/}}

\newcommand\px{\mathrel{/\mkern-5mu/}}  %平行
\newcommand\pxeq{\mathrel{\vcenter{     %平行且等于
\ialign{\hfil##\hfil\crcr
$\scriptstyle\px\!$\crcr
\noalign{\nointerlineskip\vskip1pt}$=$\crcr}}}}

%\setCJKmainfont{SimSun}       %设置西文字体为times new roman
%\setCJKsansfont{SimSun}             %设置中文字体为宋体
%\setCJKmonofont{STKaiti}
%\setsansfont{TeX Gyre Termes}            %设置typewriter family中文字体为楷体
%\setmonofont{TeX Gyre Termes}

\usetikzlibrary{calc,intersections,through,backgrounds,patterns}
\newcounter{para}
\newcommand\mypara{\par\refstepcounter{para}\thepara.\space}%设置typewriter family西文字体为times new roman
\newcommand*\circled[1]{\tikz[baseline=(char.base)]{
            \node[shape=circle,draw,inner sep=1pt] (char) {#1};}}

\newcommand{\rnum}[1]{\romannumeral #1}
\newcommand{\RNum}[1]{\uppercase\expandafter{\romannumeral #1\relax}}
\begin{document}
设抛物线 $C:y^2=2px,A(a,0),B(ab,0)$,经过 $A$的直线交 $C$于 $M,N$两点,
直线$MB,NB$与 $C$的另一交点分别为 $P,Q$,求 $PQ$与 $x$轴的交点.\\
解:设 $M(\frac{m^2}{2p},m),N(\frac{n^2}{2p},n)$\\
$k_{MN}=\frac{m-n}{\frac{m^2}{2p}-\frac{n^2}{2p}}=\frac{2p}{m+n}$\\
$l_{MN}:y=\frac{2p}{m+n}(x-\frac{m^2}{2p})+m=\frac{2p}{m+n}x+\frac{mn}{m+n}
=\frac{2px+mn}{m+n}$\\
$l$经过 $(a,0)$ 则 $mn=-2pa$\\
$K_{MD}=\frac{m}{\frac{m^2}{2p}-ab}$\\
则 $MD:y=\frac{1}{\frac{m}{2p}-\frac{ab}{m}}(x-ab)$\\
$ \left\{\begin{aligned}
 &y=\frac{1}{\frac{m}{2p}-\frac{ab}{m}}(x-ab) , \\ 
 &y^2=2px 
\end{aligned}\right.$ \\
消去 $x$ 得:$(\frac{m}{2p}-\frac{ab}{m})y=\frac{y^2}{2p}-ab
\Leftrightarrow y^2-(m-\frac{2abp}{m})y-2abp=0$\\
则方程的另一解为 $\frac{-2pab}{m}=\frac{mnb}{m}=nb$\\
故 $P(\frac{n^2b^2}{2p},nb)$,同理 $Q(\frac{m^2b^2}{2p},mb)$\\
故 $k_{PQ}=\frac{2bp^2(n-m)}{b^2(n^2-m^2)}
=\frac{2p}{b(m+n)}=\frac{-mn}{ab(m+n)}$\\
$\begin{aligned}[t]
    l_{PQ}:y&=\frac{2p}{b(m+n)}(x-\frac{n^2b^2}{2p})+nb\\ 
    &=\frac{2p}{b(m+n)}x-\frac{n^2b}{m+n}+nb\\
    &=\frac{2px-n^2b^2+n^2b^2+mnb^2}{b(m+n)}\\
    &=\frac{2p}{b(m+n)}(x-ab^2)\\
    &=\frac{-mn}{ab(m+n)}(x-ab^2)
\end{aligned}$\\
$y=0$时 $x=ab^2$,故 $PQ$经过$(ab^2,0)$\\
若设 $MN:y=k(x-a)$,则 $m,n$满足$\frac{y^2}{2p}-\frac{y}{k}-a=0$\\
则 $mn=-2pa,m+n=2pk$\\
此时 $l_{MN}:y=\frac{-2pa}{ab\cdot 2pk}(x-ab^2)=\frac{-1}{kb}(x-ab^2)$\\
故其方程与抛物线参数 $p$ 无关
\end{document}