\documentclass[class=ctexart,crop=false]{standalone}

\usepackage{amsmath,amssymb,enumitem,empheq,tkz-euclide,
diagbox,wrapfig,pgfplots}
\pgfplotsset{compat=newest}
\renewcommand\parallel{\mathrel{/\mskip-2.5mu/}}

\newcommand\px{\mathrel{/\mkern-5mu/}}  %平行
\newcommand\pxeq{\mathrel{\vcenter{     %平行且等于
\ialign{\hfil##\hfil\crcr
$\scriptstyle\px\!$\crcr
\noalign{\nointerlineskip\vskip1pt}$=$\crcr}}}}

%\setCJKmainfont{SimSun}       %设置西文字体为times new roman
%\setCJKsansfont{SimSun}             %设置中文字体为宋体
%\setCJKmonofont{STKaiti}
%\setsansfont{TeX Gyre Termes}            %设置typewriter family中文字体为楷体
%\setmonofont{TeX Gyre Termes}

\usetikzlibrary{calc,intersections,through,backgrounds,patterns}
\newcounter{para}
\newcommand\mypara{\par\refstepcounter{para}\thepara.\space}%设置typewriter family西文字体为times new roman
\newcommand*\circled[1]{\tikz[baseline=(char.base)]{
            \node[shape=circle,draw,inner sep=1.3pt] (char) {#1};}}
\begin{document}
设抛物线 $C:y^2=2px,A(a,0),B(ab,0)$,经过 $A$的直线交 $C$于 $M,N$两点,
直线$MB,NB$与 $C$的另一交点分别为 $P,Q$,求 $PQ$与 $x$轴的交点.\\
解:设 $M(\frac{m^2}{2p},m),N(\frac{n^2}{2p},n)$\\
$k_{MN}=\frac{m-n}{\frac{m^2}{2p}-\frac{n^2}{2p}}=\frac{2p}{m+n}$\\
$l_{MN}:y=\frac{2p}{m+n}(x-\frac{m^2}{2p})=\frac{2p}{m+n}x-\frac{mn}{m+n}
=\frac{2px-mn}{m+n}$\\
$l$经过 $(a,0)$ 则 $mn=2pa$\\
$K_{MD}=\frac{m}{\frac{m^2}{2p}-ab}$\\
则 $MD:y=\frac{1}{\frac{m}{2p}-\frac{ab}{m}}(x-ab)$
$$ \left\{\begin{aligned}
 &y=\frac{1}{\frac{m}{2p}-\frac{ab}{m}}(x-ab) , \\ 
 &y^2=2px 
\end{aligned}\right.$$
消去 $x$ 得:$(\frac{m}{2p}-\frac{ab}{m})y=\frac{y^2}{2p}-ab
\Leftrightarrow y^2-(m-\frac{2abp}{m})y-2abp=0$\\
$故 $A(\frac{16}{m^2},-\frac{2abp}{m})$\\
同理 $B(\frac{16}{n^2},-\frac{8}{n})$\\
故 $k_{AB}=\frac{\frac{8}{m}-\frac{8}{n}}{16(\frac{1}{n^2}-\frac{1}{m^2})}
=\frac{-1}{2(\frac{1}{m}+\frac{1}{n})}=\frac{-mn}{2(m+n)}=\frac{k}{2}$\\
$\begin{aligned}[t]
    l_{AB}&=\frac{-mn}{2(m+n)}(x-\frac{16}{m^2})-\frac{8}{m}\\ 
    &=\frac{-mn}{2(m+n)}x+\frac{8n}{m(m+n)}-\frac{8}{m}\\
    &=\frac{-mnx-16}{2(m+n)}\\
    &=\frac{k}{2}(x-4)
\end{aligned}$\\
$y=0$时 $-\frac{8}{m}\cdot\frac{2(m+n)}{mn}=x-\frac{16}{m^2}\Rightarrow x=\frac{-16}{mn}=4$\\
故  $\alpha,\beta$同是第一象限角或第二象限角,则 $\alpha-\beta\in (-\frac{\pi}{2},\frac{\pi}{2})$\\
此时 $\tan{x}$单调递增\\
$\tan(\alpha-\beta)=\frac{\tan{\alpha}-\tan{\beta}}{1+\tan{\alpha}\tan{\beta}}
=\frac{k-\frac{k}{2}}{1+\frac{k^2}{2}}=\frac{1}{\frac{2}{k}+k}$\\
显然 $k<0$时,$\tan(\alpha-\beta)<0,\alpha-\beta<0$\\
$k>0$时,$\tan(\alpha-\beta)=\frac{1}{\frac{2}{k}+k}\leqslant \frac{1}{2\sqrt{2}}$\\
$k=\sqrt{2}$时取等号,此时 $\tan(\alpha-\beta)$取得最大值,$\alpha-\beta$也取得最大值\\
此时 $m\text{满足} m^2-2\sqrt{2}m-4=0,\text{取}m=\sqrt{2}+\sqrt{6}\\
\text{则}A(8-4\sqrt{3},2\sqrt{2}-2\sqrt{6})$\\
$\text{故}k_{AB}=\frac{\sqrt{2}}{2},AB:y=\frac{\sqrt{2}}{2}(x-8+4\sqrt{3})+2\sqrt{2}-2\sqrt{6}=\frac{\sqrt{2}}{2}x-2\sqrt{2}$

\end{document}