\documentclass[class=ctexart,crop=false]{standalone}

\usepackage{amsmath,amssymb,enumitem,empheq,tkz-euclide,
diagbox,wrapfig,pgfplots}
\pgfplotsset{compat=newest}
\renewcommand\parallel{\mathrel{/\mskip-2.5mu/}}

\newcommand\px{\mathrel{/\mkern-5mu/}}  %平行
\newcommand\pxeq{\mathrel{\vcenter{     %平行且等于
\ialign{\hfil##\hfil\crcr
$\scriptstyle\px\!$\crcr
\noalign{\nointerlineskip\vskip1pt}$=$\crcr}}}}

%\setCJKmainfont{SimSun}       %设置西文字体为times new roman
%\setCJKsansfont{SimSun}             %设置中文字体为宋体
%\setCJKmonofont{STKaiti}
%\setsansfont{TeX Gyre Termes}            %设置typewriter family中文字体为楷体
%\setmonofont{TeX Gyre Termes}

\usetikzlibrary{calc,intersections,through,backgrounds,patterns}
\newcounter{para}
\newcommand\mypara{\par\refstepcounter{para}\thepara.\space}%设置typewriter family西文字体为times new roman
\newcommand*\circled[1]{\tikz[baseline=(char.base)]{
            \node[shape=circle,draw,inner sep=1.3pt] (char) {#1};}}
\begin{document}
如图,已知点 $F(1,0)$ 为抛物线 $y^2=2px(p>0)$ 的焦点,过点 $F$ 的直线
交抛物线于 $A,B$ 两点,点 $C$ 在抛物线上,使得 $\triangle ABC$ 的重心
$G$ 在 $x$ 轴上,直线 $AC$ 交 $x$ 轴于点 $Q$,且点$Q$在点 $F$ 的右侧,
记 $\triangle AFG,\triangle CQG$ 的面积分别为 $S_1,S_2$.
\begin{enumerate}[label=(\Roman*)]
    \item 求 $p$ 的值及抛物线的准线方程;
    \item 求 $\frac{S_1}{S_2}$的最小值及此时点 $G$ 的坐标.
\end{enumerate}

\rightline{\begin{tikzpicture}
      \begin{axis}[
        ticks=none,
        unit vector ratio*=1 1 1 ,
        x axis line style={name path global=xaxis}, 
        axis  lines = middle,
        xlabel = {$x$}, ylabel = {$y$},
        x label style={anchor=north},
        y label style={anchor=east},
        xmin=-1, xmax=6,
        ymin=-6, ymax=6,
        legend pos=outer north east,
        legend style={draw=none},
      ]
        \addplot +[
          name path global=Parabola,
          no markers,
          raw gnuplot,
          thick,
          black,
          empty line = jump % not strictly necessary, as this is the default behaviour in the development version of PGFPlots
        ] gnuplot {
          set xrange [0:5];
          set yrange [-5:5];
          set contour base;
          set cntrparam levels discrete 0.003;
          unset surface;
          set view map;
          set isosamples 500;
          set samples 500;
          splot y^2-4*x ;
        };
        \node[label={120:{O}},circle,fill,inner sep=1pt] (O) at (axis cs:0,0) {};
        \node[label={120:{A}},circle,fill,inner sep=1pt] (A) at (axis cs:2+3^0.5,2^0.5+6^0.5) {};
        \node[label={180:{B}},circle,fill,inner sep=1pt] (B) at (axis cs:2-3^0.5,2^0.5-6^0.5) {};
        \node[label={90:{F}},circle,fill,inner sep=1pt] (F) at (axis cs:1,0) {};
        \node[label={265:{G}},circle,fill,inner sep=1pt] (G) at (axis cs:2,0) {};
        \node[label={210:{C}},circle,fill,inner sep=1pt] (C) at (axis cs:2,-2*2^0.5) {};
        \draw (A) -- (B) --(C)--(G)--(A);
        \draw [name path=AC] (A) -- (C);
 %       \node[label={330:{Q}},circle,fill,inner sep=1pt] (Q) at(intersection 1 of xaxis and A--C)
        \path  [name intersections={of=AC and xaxis}]
        (intersection-1)  coordinate [label={330: {Q}},circle,fill,inner sep=1pt] (Q);
 %      \path [name intersections={of=xaxis and AC, by=Q}] node[dot, label= $Q$] at (Q)  {} ;
      \end{axis}
      \end{tikzpicture}}
\end{document}
