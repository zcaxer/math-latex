\documentclass[class=ctexart,crop=false]{standalone}

\usepackage{amsmath,amssymb,enumitem,empheq,tkz-euclide,
diagbox,wrapfig,pgfplots,geometry}
%\geometry{a4paper,scale=0.9}
\pgfplotsset{compat=newest}
%\usepgfplotslibrary{external}
%\tikzexternalize
\renewcommand\parallel{\mathrel{/\mskip-2.5mu/}}

\newcommand\px{\mathrel{/\mkern-5mu/}}  %平行
\newcommand\pxeq{\mathrel{\vcenter{     %平行且等于
\ialign{\hfil##\hfil\crcr
$\scriptstyle\px\!$\crcr
\noalign{\nointerlineskip\vskip1pt}$=$\crcr}}}}

%\setCJKmainfont{SimSun}       %设置西文字体为times new roman
%\setCJKsansfont{SimSun}             %设置中文字体为宋体
%\setCJKmonofont{STKaiti}
%\setsansfont{TeX Gyre Termes}            %设置typewriter family中文字体为楷体
%\setmonofont{TeX Gyre Termes}

\usetikzlibrary{calc,intersections,through,backgrounds,patterns}
\newcounter{para}
\newcommand\mypara{\par\refstepcounter{para}\thepara.\space}%设置typewriter family西文字体为times new roman
\newcommand*\circled[1]{\tikz[baseline=(char.base)]{
            \node[shape=circle,draw,inner sep=1pt] (char) {#1};}}

\newcommand{\rnum}[1]{\romannumeral #1}
\newcommand{\RNum}[1]{\uppercase\expandafter{\romannumeral #1\relax}}
\begin{document}
\begin{enumerate}[label=(\Roman*)]
    \item $a=-\frac{3}{4}$时,$f(x)=-\frac{3}{4}\ln{a}+\sqrt{1+x}$\\
    $f'(x)=-\frac{3}{4x}+\frac{1}{2\sqrt{1+x}}=\frac{2x-3\sqrt{1+x}}{4x\sqrt{1+x}}$\\
    $f'(x)=0$时 $2x=\sqrt{1+x}\Rightarrow 4x^2-9x-9=0\\
    \Rightarrow (x-3)(4x+3)=0$\\
    $x_1=3,x_2=-\frac{3}{4}$,$x>3,f'(x)>0,0<x<3,f(x)<0$\\
    故单调递增区间为 $(3,+\infty)$,单调递减区间为 $(0,3)$
    \item $a\ln{x}+\sqrt{1+x}-\frac{\sqrt{x}}{2a}\leqslant 0$\\
    注意到 带入$x=1,$得 $\frac{1}{2a}\geqslant \sqrt{2}\Rightarrow 0<a \leqslant \frac{\sqrt{2}}{4}$\\
    $a>0$时,原式等价于 $\sqrt{x}\cdot\frac{1}{a^2}-2\sqrt{x+1}\cdot\frac{1}{a}-2\ln{x}\geqslant 0$\\
    %将其视为关于 $a$ 的二次函数,其开口向上,对称轴为 $\frac{\sqrt{x+1}}{\sqrt{x}}=\sqrt{1+\frac{1}{x}}$\\
    %$x \in [\frac{1}{e^2},+\infty)$时,$\sqrt{1+\frac{1}{x}}\in (1,\sqrt{e^2+1}]$\\
    记 $t=\frac{1}{a},t\geqslant 2\sqrt{2}\\
    $则$g(t)=\sqrt{x}\cdot t^2-2\sqrt{x+1}\cdot t-2\ln{x}
    =\sqrt{x}(t-\sqrt{1+\frac{1}{x}})^2-2\ln{x}-\frac{1+x}{\sqrt{x}}$\\
    $g(t)$开口向上,其对称轴 $t=\sqrt{1+\frac{1}{x}}\leqslant 2\sqrt{2}$
    即 $x\geqslant\frac{1}{7}$ 时,$g(t)$单调递增\\
    故 $g(t)\geqslant g(2\sqrt{2})=8\sqrt{x}-4\sqrt{2}\sqrt{x+1}-2\ln{x}$\\
    记 $p(x)=4\sqrt{x}-2\sqrt{2}\sqrt{x+1}-\ln{x},x\geqslant \frac{1}{7}$\\
    $\begin{aligned}[t]
        p'(x)&=\frac{2}{\sqrt{x}}-\frac{\sqrt{2}}{\sqrt{1+x}}-\frac{1}{x}\\
        &=\frac{\sqrt{1+x}-\sqrt{2x}}{\sqrt{x(1+x)}}+\frac{\sqrt{x}-1}{x}\\
        &=\frac{1-x}{\sqrt{x(x+1)}(\sqrt{2x}+\sqrt{x+1})}+\frac{x-1}{x(\sqrt{x}+1)}\\
        &=(x-1)\frac{(\sqrt{2x}+\sqrt{x+1})(\sqrt{x+1})-\sqrt{x}(\sqrt{x}+1)}{x\sqrt{x+1}(\sqrt{x}+1)(\sqrt{2x}+\sqrt{x+1})}\\
        &=\frac{(x-1)(\sqrt{2x^2+2x}+1-\sqrt{x})}{x\sqrt{x+1}(\sqrt{x}+1)(\sqrt{2x}+\sqrt{x+1})}
    \end{aligned}$\\
    $x>1,p'(x)>0;x<1,p'(x)<0$故 $p(x)$在 $x=1$处取得极小值 $p(1)=0$\\
    故 $x\geqslant \frac{1}{7}\text{时},g(t)\geqslant g(2\sqrt{2})= 2p(x)\geqslant 0 $\\
    $\frac{1}{e^2}\leqslant x <\frac{1}{7}$时,对称轴 $\sqrt{1+\frac{1}{x}}>2\sqrt{2}$\\
    此时 $g(t)\geqslant g(\sqrt{1+\frac{1}{x}})=\sqrt{x}+\frac{1}{\sqrt{x}}-2\sqrt{(1+\frac{1}{x})(1+x)}-2\ln{x}=\frac{-(x+1+2\sqrt{x}\ln{x})}{\sqrt{x}}$\\
    记 $q(x)=x+1+2\sqrt{x}\ln{x},\frac{1}{e^2}\leqslant x<\frac{1}{7},q'(x)=1+\frac{\ln{x}}{\sqrt{x}}+\frac{2}{\sqrt{x}}>0$\\
    故 $q(x)$单调递增,$q(x)< q(\frac{1}{7})$\\
    故此时 $g(t)>g(\sqrt{1+\frac{1}{\frac{1}{7}}})=g(2\sqrt{2})\geqslant 0$ 

\end{enumerate}
\end{document}